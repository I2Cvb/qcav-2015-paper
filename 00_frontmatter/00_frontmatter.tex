\title{Style template and guidelines for SPIE Proceedings} 

%>>>> The author is responsible for formatting the 
%  author list and their institutions.  Use  \skiplinehalf 
%  to separate author list from addresses and between each address.
%  The correspondence between each author and his/her address
%  can be indicated with a superscript in italics, 
%  which is easily obtained with \supit{}.

\author{Guillaume~Lema\^{i}tre\supit{a,c}, Joan~Massich\supit{a}, Robert~Mart\'{i}\supit{c}, Jordi~Freixenet\supit{c}, Joan~C.~Vilanova\supit{d}, Paul~M.~Walker\supit{b}, Fabrice~M\'{eriaudeau}\supit{a}
\skiplinehalf
\supit{a}{\scriptsize LE2I-UMR CNRS 6306, Universit\'{e} de Bourgogne, 12 rue de la Fonderie, 71200 Le Creusot, France;} \\
\supit{b}{\scriptsize LE2I-UMR CNRS 6306, Universit\'{e} de Bourgogne, Avenue Alain Savary, 21000 Dijon, France;} \\
\supit{c}{\scriptsize ViCOROB, Universitat de Girona, Campus Montilivi, Edifici P4, 17071 Girona, Spain;} \\
\supit{d}{\scriptsize Department of Magnetic Resonance, Cl\'{i}nica Girona, Lorenzana 36, 17002 Girona, Spain}
}

%>>>> Further information about the authors, other than their 
%  institution and addresses, should be included as a footnote, 
%  which is facilitated by the \authorinfo{} command.

\authorinfo{Further author information: (Send correspondence to G.L.)\\G.L.: E-mail: guillaume.lemaitre@udg.edu}
%%>>>> when using amstex, you need to use @@ instead of @

\maketitle 

\begin{abstract}
An abstract come here.
\end{abstract}

\keywords{keyword1, keyword2, keyword3, keyword4, keyword5}

%%% Local Variables: 
%%% mode: latex
%%% TeX-master: "../master.tex"
%%% End: 