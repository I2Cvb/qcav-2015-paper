\section{INTRODUCTION}\label{sec:introduction}

On a worldwide scale, \ac{cap} has been reported as the second most frequently diagnosed men cancers accounting for 13.6\,\%~\cite{Ferlay2010}. Statistically, the estimated number of new diagnosed cases was 899,000 with no less than 258,100 estimated deaths~\cite{Ferlay2010}. In United States, aside from skin cancer, \ac{cap} was declared to be the most commonly diagnosed cancer among men, implying that around one in seven men will be diagnosed with \ac{cap} during their lifetime~\cite{Siegel2014}.

Since its introduction in mid-1980s, \ac{psa} is widely used for \ac{cap} screening~\cite{Etzioni2002} and has shown to improve early detection of \ac{cap}~\cite{Chou2011}. However, several trials conducted in Europe and United States conclude that \ac{psa} screening suffers from low specificity~\cite{Andriole2009,Hugosson2010,Schroeder2012}. Thus, current research focus on developing new screening methods to improve \ac{cap} detection. In this perspective, \Ac{mri} techniques have recently shown promising results for \ac{cap} detection. Furthermore, three different modalities are currently investigated: (i) \ac{t2w} \ac{mri}, (ii) \ac{dce} \ac{mri} and (iii) \ac{dw} \ac{mri}.

Several researches have been carried out in order to investigate the contributions of machine learning classifiers for \ac{cap} detection using the three aforementioned 3T multi-parametric \ac{mri} such as \ac{svm}~\cite{Litjens2011,Litjens2012a,Litjens2014,Liu2013,Peng2013}, probabilistic boosting tree~\cite{Viswanath2011} or probabilistic neural network~\cite{Viswanath2011}. However, these studies use different datasets and evaluation statistics to report their results leading to an impossibility to give rise to a fair comparison.

In this research, we investigate the performance of gradient boosting for \ac{cap} detection using 3T multi-parametric \ac{mri}. Two different features extraction strategies have been chosen in order to feed the classifier: (i) voxel-based and (ii) 3D texton-based. An evaluation of both strategies as well as the contribution of each modality is provided. Furthermore, the dataset used for this experimentation is part of our future benchmarking platform I2CVB available at {\tt http://visor.udg.edu/i2cvb/} and are ready for future comparisons.

%%% Local Variables: 
%%% mode: latex
%%% TeX-master: "../../master.tex"
%%% End: 