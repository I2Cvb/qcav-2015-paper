\section{RESULTS AND DISCUSSION}\label{sec:results}

%% Figure 1
\begin{figure}[t]
%%Figure1a
\subfloat[Voxel-based features]{\label{fig:voxres}
\hspace{0.8cm}
\begin{tikzpicture} [scale=.85,every node/.style={scale=0.85}]  %[scale=.68]

\def\labels{
{\color{green}$V_{1}$}, 
{\color{green}$V_{2}$}, 
{\color{green}$V_{3}$},  
{\color{green}$V_{4}$},
{\color{green}$V_{5}$},
{\color{green}$V_{6}$},
{\color{green}$V_{7}$},
{\color{green}$V_{8}$}
 }

\def\reward{74.1,67.8,71.3,67.8,73.4,74.7,75.1,76.6}
\def\spec{42.1,54.6,69.8,57.6,69.4,69.3,69.9,73.4}
\def\cZoom{2.5} 
\def\percentageLabelAngle{10}
\def\nbeams{8}
\pgfmathsetmacro\beamAngle{(360/\nbeams)}
\pgfmathsetmacro\halfAngle{(180/\nbeams)}

\pgfmathsetmacro\globalRotation{\halfAngle}

\foreach \n  [count=\ni] in \labels
{
\pgfmathsetmacro\cAngle{{(\ni*(360/\nbeams))+\globalRotation}}
\draw(\cAngle:{\cZoom*1.00})  node[fill=white] {\n};
\draw [thin] (0,0) -- (\cAngle:{\cZoom*0.9}) ;

}

%% draw the % rings 
\foreach \x in {12.5,25, ...,100} 
\draw [thin,color=gray!50] (0,0) circle [radius={\cZoom*\x/110}];

\foreach \x in {50,75,100}
{ 
     \draw [thin,color=black!50] (0,0) circle [radius={\cZoom/110*\x}];
     \foreach \a in {0, 180} \draw ({\percentageLabelAngle+\a}:{\cZoom*0.01*\x}) node  [inner sep=0pt,outer sep=0pt,fill=white,font=\fontsize{3}{3.5}\selectfont]{$\x\%$};
}

%% draw the path of the percentages
\def\aux{{\reward}}
\pgfmathsetmacro\origin{\aux[\nbeams-1]} 
\draw [blue, thick] (\globalRotation:{\cZoom*\origin/110}) \foreach \n  [count=\ni] in \reward { -- ({(\ni*(360/\nbeams))+\globalRotation}:{\cZoom*\n/110}) } ;

\def\auxx{{\spec}}
\pgfmathsetmacro\origin{\auxx[\nbeams-1]} 
\draw [red, thick] (\globalRotation:{\cZoom*\origin/110}) \foreach \n  [count=\ni] in \spec { -- ({(\ni*(360/\nbeams))+\globalRotation}:{\cZoom*\n/110}) };


\foreach \n [count=\ni] in \reward 
{
  \pgfmathsetmacro\cAngle{{(\ni*(360/\nbeams))+\globalRotation}}
  \pgfmathsetmacro\nreward{\aux[\ni-1]}
  \pgfmathsetmacro\nspec{\auxx[\ni-1]}
  \draw (\cAngle:{\cZoom*1.36}) node[align=center] {{\color{blue}\nreward $\%$ } \\{ \color{red}\nspec $\%$}};
};

%%%% draw the domain of each class 
\def\puta{3/0/{Mono-parametric},
  5/3/{Multi-parametric}}
\foreach \numElm/\contadorQueNoSeCalcular/\name [count=\ni] in \puta
 {

   \pgfmathsetmacro\initialAngle{(\contadorQueNoSeCalcular*\beamAngle)+\halfAngle+\globalRotation}
   \pgfmathsetmacro\finalAngle  {((\numElm+\contadorQueNoSeCalcular)*\beamAngle)+\halfAngle+\globalRotation}
   \pgfmathsetmacro\l  {\cZoom*1.5+.3pt}
   \draw (\initialAngle:{\cZoom*1.6}) -- (\initialAngle:{\cZoom*1.1});
   \draw [ |<->|,>=latex] (\initialAngle:\l) arc (\initialAngle:\finalAngle:\l) ;     
   \pgfmathsetmacro\r  {\cZoom*1.5+.45pt}
   {\draw [decoration={text along path,  text={\name},text align={center}},decorate] (\finalAngle:\r) arc (\finalAngle:\initialAngle:\r);} 
    
  }  
\end{tikzpicture} 
}\hfill
%%Figure1b
\subfloat[3D texton-based features]{\label{fig;texres}
\begin{tikzpicture} [scale=.85,every node/.style={scale=0.85}]  %[scale=.68]

\def\labels{
{\color{green}$T_{1}$}, 
{\color{green}$T_{2}$}, 
{\color{green}$T_{3}$},  
{\color{green}$T_{4}$},
{\color{green}$T_{5}$},
{\color{green}$T_{6}$},
{\color{green}$T_{7}$},
{\color{green}$T_{8}$}
 }

\def\reward{74.6,75.7,82.7,78.2,84.7,84.6,85.3,86.9}
\def\spec{60.6,64.1,81.6,67.3,81.2,82.2,81.4,84.6}
\def\cZoom{2.5} 
\def\percentageLabelAngle{10}
\def\nbeams{8}
\pgfmathsetmacro\beamAngle{(360/\nbeams)}
\pgfmathsetmacro\halfAngle{(180/\nbeams)}

\pgfmathsetmacro\globalRotation{\halfAngle}

\foreach \n  [count=\ni] in \labels
{
\pgfmathsetmacro\cAngle{{(\ni*(360/\nbeams))+\globalRotation}}
\draw(\cAngle:{\cZoom*1.00})  node[fill=white] {\n};
\draw [thin] (0,0) -- (\cAngle:{\cZoom*0.9}) ;

}

% draw the % rings 
\foreach \x in {12.5,25, ...,100} 
\draw [thin,color=gray!50] (0,0) circle [radius={\cZoom*\x/110}];

\foreach \x in {50,75,100}
{ 
     \draw [thin,color=black!50] (0,0) circle [radius={\cZoom/110*\x}];
     \foreach \a in {0, 180} \draw ({\percentageLabelAngle+\a}:{\cZoom*0.01*\x}) node  [inner sep=0pt,outer sep=0pt,fill=white,font=\fontsize{3}{3.5}\selectfont]{$\x\%$};
}


% draw the path of the percentages
\def\aux{{\reward}}
\pgfmathsetmacro\origin{\aux[\nbeams-1]} 
\draw [blue, thick] (\globalRotation:{\cZoom*\origin/110}) \foreach \n  [count=\ni] in \reward { -- ({(\ni*(360/\nbeams))+\globalRotation}:{\cZoom*\n/110}) } ;


\def\auxx{{\spec}}
\pgfmathsetmacro\origin{\auxx[\nbeams-1]} 
\draw [red, thick] (\globalRotation:{\cZoom*\origin/110}) \foreach \n  [count=\ni] in \spec { -- ({(\ni*(360/\nbeams))+\globalRotation}:{\cZoom*\n/110}) };

% label all the percentags
\foreach \n [count=\ni] in \reward 
{
  \pgfmathsetmacro\cAngle{{(\ni*(360/\nbeams))+\globalRotation}}
  \pgfmathsetmacro\nreward{\aux[\ni-1]}
  \pgfmathsetmacro\nspec{\auxx[\ni-1]}
  \draw (\cAngle:{\cZoom*1.36}) node[align=center] {{\color{blue}\nreward $\%$ } \\{ \color{red}\nspec $\%$}};
};

%%% draw the domain of each class 
\def\puta{3/0/{Mono-parametric},
  5/3/{Multi-parametric}}
\foreach \numElm/\contadorQueNoSeCalcular/\name [count=\ni] in \puta
 {
   \pgfmathsetmacro\initialAngle{(\contadorQueNoSeCalcular*\beamAngle)+\halfAngle+\globalRotation}
   \pgfmathsetmacro\finalAngle  {((\numElm+\contadorQueNoSeCalcular)*\beamAngle)+\halfAngle+\globalRotation}
   \pgfmathsetmacro\l  {\cZoom*1.5+.3pt}
   \draw (\initialAngle:{\cZoom*1.6}) -- (\initialAngle:{\cZoom*1.1});
   \draw [ |<->|,>=latex] (\initialAngle:\l) arc (\initialAngle:\finalAngle:\l) ;     
   \pgfmathsetmacro\r  {\cZoom*1.5+.45pt}
   {\draw [decoration={text along path,  text={\name},text align={center}},decorate] (\finalAngle:\r) arc (\finalAngle:\initialAngle:\r);}     
  }  
\end{tikzpicture}
}
\caption{Comparison between the combination of features introduced in Table~\ref{tab:conc} in terms of sensitivity and specificity are illustrated in {\color{blue}blue} and {\color{red}red}, respectively.} 
\label{fig:result}
\end{figure}

%%% Local Variables: 
%%% mode: latex
%%% TeX-master: "../../master.tex"
%%% End: 

The classification results obtained are depicted in Fig.\,\ref{fig:result}. The best classification performances are achieved the 3D texton-based extraction strategy and a combination of the three different modalities and the zonal information. The sensitivity and specificity obtained are 86.9\% and 84.6\%, respectively.

Analyzing the classification outcomes of each single modality, the \ac{adc} map is the most discriminative feature with superior performances than the combination of \ac{t2w} \ac{mri} and \ac{dce} \ac{mri} together. However, the two latter mentioned modalities provide relevant information since that the combination of the three of them enhances the reported sensitivity and specificity. 

Integrating information about the prostate zones (i.e., \ac{pz} and \ac{cg}) boosts the classification performances. More precisely, this feature allows to improve greatly the specificity and slightly the sensitivity. 

In overall, the 3D texton-based strategy leads to better classification results compared with the voxel-based strategy for all the mono and multi parametric combinations experimented. Thus, integrating spatial information about the neighborhood of a given voxel should lead to drastic improvements.

%%% Local Variables: 
%%% mode: latex
%%% TeX-master: "../../master.tex"
%%% End: 