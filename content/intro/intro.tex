\section{INTRODUCTION}
\label{sec:intro}  % \label{} allows reference to this section

This document shows the desired format and appearance of a manuscript prepared for the Proceedings of the SPIE.\footnote{The basic format was developed in 1995 by Rick Herman (SPIE) and Ken Hanson (Los Alamos National Lab.).} It is prepared using LaTeX2e\cite{Lamport94} with the class file {\tt spie.cls}.  The LaTeX source file used to create this document is {\tt article.tex}, which contains important formatting information embedded in it.  These files are available on the Internet at {\tt http://home.lanl.gov/kmh/spie/}.  The font used throughout is the LaTeX default font, Computer Modern Roman, which is equivalent to the Times Roman font available on many systems.  If this font is not available, use a similar serif font.  Normal text has a font size of 10 points\footnote{Font sizes are specified in points, abbreviated pt., which is a unit of length.  One inch = 72.27 pt.; one cm = 28.4 pt.} for which the actual height of a capital E is about 2.4 mm (7 pt.) and the line-to-line spacing is about 4.2 mm (12 pt.).  The font attributes for other parts of the manuscript, summarized in Table~\ref{tab:fonts}, are described in the following sections.  Normal text should be justified to both the left and right margins.  Appendix~\ref{sec:latex} has information about PostScript fonts.

To be properly reproduced in the Proceedings, all text and figures must fit inside a rectangle 6.75-in.\ wide by 8.75-in.\ high or 17.15 cm by 22.23 cm.  The text width and height are set in {\tt spie.cls} to match this requirement.
%% This table is carefully placed in the source file to make 
%% it appear at bottom of page, but above the footnotes.  
%% Use of [h] in following command forces table to appear "here".
\begin{table}[h]
\caption{Fonts sizes to be used for various parts of the manuscript.  All fonts are Computer Modern Roman or an equivalent.  Table captions should be centered above the table.  When the caption is too long to fit on one line, it should be justified to the right and left margins of the body of the text.} 
\label{tab:fonts}
\begin{center}       
\begin{tabular}{|l|l|} %% this creates two columns
%% |l|l| to left justify each column entry
%% |c|c| to center each column entry
%% use of \rule[]{}{} below opens up each row
\hline
\rule[-1ex]{0pt}{3.5ex}  Article title & 16 pt., bold, centered  \\
\hline
\rule[-1ex]{0pt}{3.5ex}  Author names and affiliations & 12 pt., normal, centered   \\
\hline
\rule[-1ex]{0pt}{3.5ex}  Section heading & 11 pt., bold, centered (all caps)  \\
\hline
\rule[-1ex]{0pt}{3.5ex}  Subsection heading & 11 pt., bold, left justified  \\
\hline
\rule[-1ex]{0pt}{3.5ex}  Sub-subsection heading & 10 pt., bold, left justified  \\
\hline
\rule[-1ex]{0pt}{3.5ex}  Normal text & 10 pt., normal  \\
\hline
\rule[-1ex]{0pt}{3.5ex}  Figure and table captions & \, 9 pt., normal \\
\hline
\rule[-1ex]{0pt}{3.5ex}  Footnote & \, 9 pt., normal \\
\hline 
\end{tabular}
\end{center}
\end{table} 
The text should begin 1.00 in.\ or 2.54 cm from the top of the page.  The right and left margins should be 0.875~in.\ or 2.22 cm for US letter-size paper (8.5 in.\ by 11 in.) or 1.925 cm for A4 paper (210 mm by 297 mm) to horizontally center the text on the page.  See Appendix~\ref{sec:latex} for guidance regarding paper-size specification. 

Authors are encouraged to follow the principles of sound technical writing, as described in Refs.~\citenum{Alred03} and \citenum{Perelman97}, for example.  Many aspects of technical writing are addressed in the {\em AIP Style Manual}, published by the American Institute of Physics.  It is available on line at {\tt http://www.aip.org/pubservs/style/4thed/toc.html} or {\tt http://public.lanl.gov/kmh/AIP\verb+_+Style\verb+_+4thed.pdf}. A spelling checker is helpful for finding misspelled words. 

An author may use this LaTeX source file as a template by substituting his/her own text in each field.  This document is not meant to be a complete guide on how to use LaTeX.  For that, refer to books on LaTeX usage, such as the definitive work by Lamport\cite{Lamport94} or the very useful compendium by Mittelbach et al.\cite{Mittelbach04}

%%% Local Variables: 
%%% mode: latex
%%% TeX-master: "../../master.tex"
%%% End: \section{introduction}
