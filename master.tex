%  article.tex (Version 3.3, released 19 January 2008)
%  Article to demonstrate format for SPIE Proceedings
%  Special instructions are included in this file after the
%  symbol %>>>>
%  Numerous commands are commented out, but included to show how
%  to effect various options, e.g., to print page numbers, etc.
%  This LaTeX source file is composed for LaTeX2e.

%  The following commands have been added in the SPIE class 
%  file (spie.cls) and will not be understood in other classes:
%  \supit{}, \authorinfo{}, \skiplinehalf, \keywords{}
%  The bibliography style file is called spiebib.bst, 
%  which replaces the standard style unstr.bst.  

\documentclass[letter]{spie}  %>>> use for US letter paper
%%\documentclass[a4paper]{spie}  %>>> use this instead for A4 paper
%%\documentclass[nocompress]{spie}  %>>> to avoid compression of citations
%% \addtolength{\voffset}{9mm}   %>>> moves text field down
%% \renewcommand{\baselinestretch}{1.65}   %>>> 1.65 for double spacing, 1.25 for 1.5 spacing 

%% Latex documents that need direct input

% %  The following command loads a graphics package to include images 
% %  in the document. It may be necessary to specify a DVI driver option,
% %  e.g., [dvips], but that may be inappropriate for some LaTeX 
% %  installations.
% \usepackage{epsf,graphicx}
% \usepackage{epstopdf}
% \usepackage{caption}
% \usepackage{subcaption}

% % Clever cross referencing. Using cleverref, instead of writting 
% % figure~\ref{...} or equation~\ref{...}, only \cref{...} is required.
% % The package interprates the references and introduces the figure, fig.,
% % equation, eq., etc keywords. \Cref forces first letter capital. 
% \usepackage{cleveref}

% % In order to include files without having a clear page using \include*, 
% % the newclude package is required
\usepackage{newclude}

% % Required for acronyms
% \usepackage{acro}

% % For multirow environment
\usepackage{multirow}

% % To use subscript environment
\usepackage{fixltx2e}

% % Packages and defintion for tick and cross
% \usepackage{pifont}
% \newcommand{\cmark}{\large \color{green!60!black!80}\ding{51}}
% \newcommand{\xmark}{\large \color{red!60!black!80}\ding{55}}
% \newcommand{\cmarksmall}{\color{green!60!black!80}\ding{51}}
% \newcommand{\xmarksmall}{\color{red!60!black!80}\ding{55}}

% %% In order to draw some graphs
% \usepackage{tikz,xifthen}
% \usepackage{tikz-qtree}
% \usetikzlibrary{decorations.pathmorphing} % noisy shapes
% \usetikzlibrary{fit}                                            % fitting shapes to coordinates
% \usetikzlibrary{backgrounds}                                    % drawing the background after the foreground
% \usetikzlibrary{shapes,arrows,shadows}
% \usetikzlibrary{calc,decorations.pathreplacing,decorations.markings,positioning}
% \usetikzlibrary{snakes,decorations.text,shapes,patterns}

%% The amssymb package provides various useful mathematical symbols
\usepackage{amssymb}

%% The amsthm package provides extended theorem environments
%\usepackage{amsthm}

%% amsmath for math environment
\usepackage{amsmath}

\DeclareMathOperator*{\argmin}{arg\,min}
\DeclareMathOperator*{\argmax}{arg\,max}
\DeclareMathOperator*{\sign}{sign}

% to break equation
%\usepackage{mathpazo}
%\usepackage{mathptmx}
%\usepackage[mathpazo]{flexisym}
%\usepackage{breqn}

%% For clever reference
%\usepackage{cleveref}

%% color package
\usepackage{color}

%% figure package
\usepackage{epsf,graphicx}
\usepackage{epstopdf}
\usepackage{subfig}     
%\usepackage{transparent}

%% New environment to have some indent inside enumerate environment
\usepackage{enumitem}

%% To create acronym for proper glossary
\usepackage{acro}
\usepackage[acronym,nomain]{glossaries}

%% To number the line in the article
\usepackage{lineno}

%% Environment to include table with notes
\usepackage{array}
\usepackage{threeparttable}
\usepackage{multirow}

%% In order to change size of margin
\usepackage{geometry}
\usepackage{changepage}

%% Colorpackage for table
\usepackage{colortbl}
\usepackage{tabularx}
\usepackage{arydshln}

%% To use URL referencing
%\usepackage[draft,hidelinks]{hyperref}

%% In order to draw some graphs
\usepackage{tikz,xifthen}
\usepackage{tikz-qtree}
\usetikzlibrary{decorations.pathmorphing} % noisy shapes
\usetikzlibrary{fit}                                            % fitting shapes to coordinates
\usetikzlibrary{backgrounds}                                    % drawing the background after the foreground
\usetikzlibrary{shapes,arrows,shadows}
\usetikzlibrary{calc,decorations.pathreplacing,decorations.markings,positioning}
\usetikzlibrary{snakes,decorations.text,shapes,patterns}
%\usepackage{scalefnt,lmodern,booktabs}

%% Paxkage for cross and tick symbols
\usepackage{pifont}
%\usepackage{amsfonts} %

%% Paxkage for cross and tick symbols
\usepackage{pifont}
\newcommand{\cmark}{\large \color{green!60!black!80}\ding{51}}
\newcommand{\xmark}{\large \color{red!60!black!80}\ding{55}}

% code listing packages
\usepackage{algorithm}
\usepackage{algpseudocode}


%other packages
\usepackage{framed}
\usepackage{quoting}
\usepackage[textwidth=3.7cm]{todonotes}

\usepackage{fixltx2e}

%Import the natbib package and sets a bibliography  and citation styles
%\usepackage{natbib}
%%\bibliographystyle{abbrvnat}
%%\usepackage{keyval}
%\setcitestyle{authoryear,open={((},close={))}}
%\bibpunct{[}{]}{,}{a}{}{;}
%%\setcitestyle{square,aysep={},yysep={;}}

\usepackage{standalone}
\usepackage{tikz,enumitem,setspace,kantlipsum}
\usetikzlibrary{shapes,positioning,arrows,calc,intersections}
\usepackage{lipsum}

\usepackage[official]{eurosym}
\usepackage{pbox}        % contains the latex packages

\title{A boosting approach for prostate cancer detection using multi-parametric MRI} 

%>>>> The author is responsible for formatting the 
%  author list and their institutions.  Use  \skiplinehalf 
%  to separate author list from addresses and between each address.
%  The correspondence between each author and his/her address
%  can be indicated with a superscript in italics, 
%  which is easily obtained with \supit{}.

% \author{Anna A. Author1\supit{a} and Barry B. Author2\supit{b}
% \skiplinehalf
% \supit{a}Affiliation1, Address, City, Country; \\
% \supit{b}Affiliation2, Address, City, Country
% }

\author{Guillaume~Lema\^{i}tre\supit{a,c} and
                Joan~Massich\supit{a} and
                Robert~Mart\'{i}\supit{c} and
                Jordi~Freixenet\supit{c} and
                Joan~C.~Vilanova\supit{d} and
                Paul~M.~Walker\supit{b} and
                D\'esir\'e~D.~Sidib\'e\supit{a} and
                Fabrice~M\'{eriaudeau}\supit{a}
  \skiplinehalf
  \supit{a}{\small LE2I-UMR CNRS 6306, Universit\'{e} de Bourgogne, 12 rue de la Fonderie, 71200 Le Creusot, France;} \\
  \supit{b}{\small LE2I-UMR CNRS 6306, Universit\'{e} de Bourgogne, Avenue Alain Savary, 21000 Dijon, France;} \\
  \supit{c}{\small ViCOROB, Universitat de Girona, Campus Montilivi, Edifici P4, 17071 Girona, Spain;}\\
  \supit{d}{\small Department of Magnetic Resonance, Clinica Girona, Lorenzana 36, 17002 Girona, Spain} \\
}

%>>>> Further information about the authors, other than their 
%  institution and addresses, should be included as a footnote, 
%  which is facilitated by the \authorinfo{} command.

\authorinfo{Further author information: (Send correspondence to G.L.)\\G.L.: E-mail: guillaume.lemaitre@udg.edu}
%%>>>> when using amstex, you need to use @@ instead of @

%%% Local Variables: 
%%% mode: latex
%%% TeX-master: "../../master.tex"
%%% End:              % contains the Title and Autor info
\input{latex/filesystem/fileSetup.tex}      % contains package and variables init.
\input{content/acronym_definition.tex}      % contains the acronims 

%% Select inputing only one part of the document
%\includeonly{content/intro/intro}   % the file wihtout .tex
%\includeonly{content/other/other_content}
 
\begin{document} 
\maketitle 

\begin{abstract}
Prostate cancer has been reported as the second most frequently diagnosed men cancers in the world. In the last decades, new imaging techniques based on MRI have been developed in order to improve the diagnosis task of radiologists. In practise, diagnosis can be affected by multiple factors reducing the chance to detect potential lesions. Computer-aided detection and computer-aided diagnosis have been designed to answer to these needs and provide help to radiologists in their daily duties. In this study, we proposed an automatic method to detect prostate cancer from a per voxel manner using 3T multi-parametric MRI and a gradient boosting classifier. The best performances are obtained using all multi-parametric information as well as zonal information. The sensitivity and specificity obtained are 86.9\% and 84.6\%, respectively.
\end{abstract}

\keywords{Gradient boosting, multi-parametric MRI, prostate cancer, computer-aided diagnosis}

%% Incldue the content without .tex extension
\section{INTRODUCTION}\label(sec:introduction)


%%% Local Variables: 
%%% mode: latex
%%% TeX-master: "../../master.tex"
%%% End:    % the file wihtout .tex
\section{MATERIAL AND METHODS}\label{sec:methodology}


\subsection{Data}\label{subsec:data}

Here, right a paragraph on how the data were acquired and which features were extracted

\subsection{Classification framework}

\subsubsection{Feature extraction strategies}

\begin{itemize}
\item voxel-based
\item 3d-texton-based
\end{itemize}

\subsubsection{Gradient boosting}

\subsection{Validation model}

k-cross validation

%%% Local Variables: 
%%% mode: latex
%%% TeX-master: "../../master.tex"
%%% End: 
\section{RESULTS}\label{sec:results}

%%% Local Variables: 
%%% mode: latex
%%% TeX-master: "../../master.tex"
%%% End: 
%\section{DISCUSSION}\label{sec:discussion}

Discuss the results.
\begin{itemize}
\item Single modality is not working as good as multi-parametric
\item Which single modality is better.
\item What is the increase of the zone information.
\item Voxel-based vs texton-based
\end{itemize}

%%% Local Variables: 
%%% mode: latex
%%% TeX-master: "../../master.tex"
%%% End: 
\section{CONCLUSION}\label{sec:conclusion}


%%% Local Variables: 
%%% mode: latex
%%% TeX-master: "../../master.tex"
%%% End: 

\acknowledgments     %>>>> equivalent to \section*{ACKNOWLEDGMENTS}       

Guillaume Lema\^itre was supported by the Generalitat de Catalunya (grant nb. FI-DGR2012) and partly by the Mediterranean Office for Youth (grant nb. 2011/018/06). We would like also to thank the Clinica Girona (Catalunya,
Espanya) and the Centre Hospitalier of Dijon (France) for providing the MRI images used in this research.


\section*{BIOGRAPHY}

Guillaume Lemaitre received an Erasmus Mundus MSc in Vision and Robotic (ViBOT) from the Heriot-Watt University, Universit\'e de Bourgone and Universitat de Girona as well as a MSc in Business Innovation and Technology Management (BITM) from the Universitat de Girona. He is currently carrying out a joint PhD at the Universit\'e de Bourgogne and Universitat de Girona focusing on the development of computer-aided diagnosis for prostate cancer. 

%%% Local Variables: 
%%% mode: latex
%%% TeX-master: "../../master"
%%% End: 


%%%%%%%%%%%%%%%%%%%%%%%%%%%%%%%%%%%%%%%%%%%%%%%%%%%%%%%%%%%%%
%%%%% References %%%%%

\bibliography{./content/literature_review}   %>>>> bibliography data in report.bib
\bibliographystyle{spiebib}   %>>>> makes bibtex use spiebib.bst

\end{document} 
